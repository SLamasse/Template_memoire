\documentclass{MemoireMimo}



%% on doit lui dire où se trouve la bibliographie 
\addbibresource{Bibliographie.bib}

\makeindex
\makeatletter


%%%%%%%%%%%%%%%%%%%%%%%%%%%
%Profondeur de \subsubsection = 3 pour le sommaire%
%%%%%%%%%%%%%%%%%%%%%%%%%%%
\setcounter{tocdepth}{3} % Dans la table des matieres
\setcounter{secnumdepth}{3} % Avec un numero.




%      Renseignements pour la page de titre 	%
\title{Mon oeuvre fabuleuse}	%% Le titre de l'oeuvre
\author{Le nom de l'artiste}	%% Le nom de l'artiste
\date{jj mois aaaa}		%% 02 mars 3012
\tuteur{Prénom et Nom}
\MaitreApprentissage{Monsieur}
\jury{
	Prénom \textsc{Nom}  (Professeur, Université Paris ??) & examinateur  \\
        Prénom \textsc{Nom} (statut, Entreprise) & maître d'apprentissage \\
}

% dédicace à votre directeur de master 

\dedicate{\og \fg } 

%      Résumé de dernière page 	%
\titleFR{Titre en français}
\titleEN{Titre en anglais }
\abstractFR{Mon résumé en français}
\abstractEN{Mon résumé en anglais}
\keywordsFR{quelques mots clefs en français}
\keywordsEN{Les mêmes mots clefs mais en anglais}




%%%%%%%%%%%%%%%%%%%%%%
%%%
%%%
%%%
%%%       Entrer dans le document
%%%
%%%
%%%
%%%%%%%%%%%%%%%%%%%%%%


\begin{document}

% Insérer la page de présentation 
 \maketitle

%%% Il faut peut-être afficher la table des matières ?
  \tableofcontents

% \mainmatter



%%%%%% Insertion des fichiers à compiler
%    	       	\include{remerciements}
%   	       	\include{abreviations} 	
%            	\include{introduction}


  
%%%%%%%%%%%%% Traitement des annexes
%%% liste des tableaux
 % \listoftables{}

%%% La liste des figures
%  \listoffigures{}


%%% un index des personnes
% \printindex

 
%%%%%%%%%%%%%%%%% 
% début de bibliographie
 % on utilise biber 
%%%%%%%%%%%%%%%%%	
\nocite{*}
\printbibliography[title={Une bibliographie belle et précise}]

\abstractpage
\end{document}

